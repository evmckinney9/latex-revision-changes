\documentclass{article}
\usepackage{reviewercomments} % Use the custom package

\title{Usage Example of the Reviewer Comments Package}
\author{The Author}

\begin{document}
\maketitle

\section{Introduction}

In the realm of scientific discovery, the role of computational models \reviewer{A}{1}{cannot be overstated.} 
These models \reviewer[false]{A}{1}{provide the backbone for simulation, data analysis, and theory-building.} 
Moreover, they enable researchers to tackle complex problems that are otherwise intractable through analytical means. Additionally, computational models have found applications across a wide range of scientific disciplines, from physics to social sciences.

However, as with any tool, there are challenges and limitations \reviewer{B}{2}{that need to be addressed, such as computational efficiency and model interpretability.} 
The aim of this study is to delve into these challenges and propose viable solutions. The goal is to contribute to the broader conversation about how computational models can be optimized for both performance and interpretability.

In recent years, there has been a surge of interest in machine learning methods for scientific discovery. These methods offer \reviewer{C}{1}{unprecedented opportunities for data-driven hypothesis generation,} but also raise important questions about validity and generalizability. The balance between these opportunities and challenges is an active area of research.

Our work builds upon the foundational theories proposed in the early 2000s, but introduces \reviewer{D}{1}{novel methodologies} that have the potential to revolutionize how we approach scientific problems. These methodologies are the cornerstone of this paper and will be discussed in detail in the subsequent sections.

Finally, the environmental impact of computational research is a growing concern. \reviewer{E}{1}{We advocate for responsible computing practices that minimize energy consumption.} This is not just an ethical imperative but also has implications for the long-term sustainability of computational research.

\end{document}
